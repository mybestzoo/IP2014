\documentclass[12pt]{iopart}

\usepackage{amssymb,amsfonts}
\expandafter\let\csname equation*\endcsname\relax
\expandafter\let\csname endequation*\endcsname\relax
\usepackage{amsmath}
%\usepackage{iopams}
\usepackage{graphicx,float}
\usepackage[caption = false]{subfig}
\usepackage{subcaption}
%\usepackage{setspace}
%\usepackage{cite}
%\usepackage{indentfirst}
\usepackage{color,soul}

\newtheorem{theorem}{Theorem}
\newtheorem{lemma}{Lemma}
\newtheorem{corollary}{Corollary}
\newenvironment{proof}
{\par\noindent{\bf Proof}}
{\hfill$\scriptstyle\blacksquare$}

\begin{document}
	
	To visualize the results we take Gaussian function $f(x)=\frac{1}{\sqrt{\pi}}e^\frac{-|x|^2}{2}$, which belongs to class $W$ for $d=2$, $k=1$, $\alpha = 2$. Its Radon transform is independent of $\theta$ (also a Gaussian), that we denote $Rf(s)$. We add the error in form 
	
	$$e(s) = \begin{cases}
a*\sin(s)+b*cos(s), & s\in[-1,1] \\
0 & s\notin[-1,1]
\end{cases}
$$
where $a$ and $b$ are normally distributed numbers with mean $0$ and variance $1$. The inaccurate approximation of the Radon transform in this case is presented by $g(s)=Rf(s)+\delta \cdot e(s)/\|e\|_{L_2(Z)}$. We apply the optimal method to reconstruct function $f$ for $\delta=10$ and compare it to the reconstruction by the "natural" method, which assumes $a(\sigma)=1$.

	\begin{figure}[h]
		\subfloat[Contour plot of Gaussian function]{\includegraphics[width = .5\linewidth]{Gaussian.png}}
		\subfloat[Slice of the Gaussian function and its recovery.]{\includegraphics[width = .5\linewidth]{Slice.png}}
		\caption{Original image and recovery results by the "natural" and optimal methods.}
		\label{some example}
	\end{figure}
	
	\begin{figure}[h]
		{\includegraphics[width = 1\linewidth]{Recon.png}}
		\caption{Contour plots of recovery results by the "natural" (left) and optimal (right) methods.}
		\label{some example}
	\end{figure}
	
	

\end{document}
